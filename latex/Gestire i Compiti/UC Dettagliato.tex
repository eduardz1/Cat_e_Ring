
\section*{Informazioni generali}
\textbf{Nome caso d'uso}{: Gestire i Compiti della Cucina}\newline
\textbf{Portata}{: Sistema}\newline
\textbf{Livello}{: Obiettivo utente}\newline
\textbf{Attore primario}{: Cuoco}\newline
\textbf{Parti Interessate}\newline
\textbf{Pre-condizioni}{: L'attore deve essere autenticato come Cuoco}\newline
\textbf{Garanzie di successo o post-condizioni}{: La procedura di cucina è salvata}

\section*{Scenario Principale di Successo}\addcontentsline{toc}{section}{Scenario Principale di Successo}
\def\arraystretch{1.55}
\begin{table}[H]\centering
    \small
    \begin{tabular}{|c|p{7cm}|p{6.5cm}|}
        \hline\bfseries \# & \bfseries Attore                                                                                                                                              & \bfseries Sistema                                                                                                                                               \\\hline
        1                  & Genera il foglio riepilogativo per un servizio di un evento (di cui ha ricevuto l’incarico)                                                                   & Registra il nuovo foglio riepilogativo per il servizio specificato in modo che contenga tutte le preparazioni e ricette previste dal menù associato al servizio, se esiste già un foglio riepilogativo per quel servizio lo restituisce \\\hline
        \multicolumn{3}{|r|}{\textit{Se desidera prosegue con il passo 2 altrimenti termina il caso d’uso}}                                                                                                                                                                                                                                                  \\\hline
        2                  & Opzionalmente aggiunge preparazioni e ricette all’elenco delle cose da fare                                                                                   & Registra l’aggiunta di preparazioni e ricette al foglio riepilogativo                                                                                           \\\hline
        3                  & Opzionalmente ordina l’elenco                                                                                                                                 & Registra l’ordinamento del foglio riepilogativo                                                                                                                 \\\hline
        \multicolumn{3}{|r|}{Se vuole lavorare su più fogli riepilogativi ripete dal passo 1}                                                                                                                                                                                                                                                                \\
        \multicolumn{3}{|r|}{Se non vuole assegnare dei compiti torna al passo 2 o termina il caso d’uso}                                                                                                                                                                                                                                                    \\\hline
        4                  & Opzionalmente, consulta tabellone di turni                                                                                                                    & Fornisce tabellone dei turni                                                                                                                                    \\\hline
        5                  & Assegna un compito specificando cosa (ricetta/preparazione), quando (turno) e opzionalmente chi (cuoco)                                                       & Registra nel foglio riepilogativo un nuovo assegnamento e aggiorna il tabellone dei turni di conseguenza                                                        \\\hline
        6                  & Opzionalmente, indica una stima del tempo richiesto per lo svolgimento del compito appena assegnato, e la quantità/porzioni preparate in un dato assegnamento & Registra le informazioni fornite sul tabellone dei turni e sul foglio riepilogativo                                                                             \\\hline
        \multicolumn{3}{|r|}{Ripete dal passo 4 sinché non è soddisfatto}                                                                                                                                                                                                                                                                                    \\
        \multicolumn{3}{|r|}{Se desidera torna al passo 2, se no termina il caso d’uso}                                                                                                                                                                                                                                                                      \\\hline
    \end{tabular}
\end{table}

\addcontentsline{toc}{subsection}{Estensioni 1}
\begin{table}[H]\centering\caption*{Estensione 1a}
    \small
    \begin{tabular}{|c|p{7cm}|p{6.23cm}|}
        \hline\bfseries \# & \bfseries Attore                                                                                               & \bfseries Sistema                          \\\hline
        1a.1               & Parte da un foglio riepilogativo esistente (tra quelli dei servizi degli eventi di cui ha ricevuto l’incarico) & Fornisce un foglio ripeligoativo esistente \\\hline
        \multicolumn{3}{|r|}{Se desidera prosegue al passo due altrimenti temrina il caso d'uso}                                                                                         \\\hline
    \end{tabular}
\end{table}

\addcontentsline{toc}{subsection}{Estensioni 2}
\begin{table}[H]\centering\caption*{Estensione 2a}
    \small
    \begin{tabular}{|c|p{7cm}|p{6.23cm}|}
        \hline\bfseries \# & \bfseries Attore                                                             & \bfseries Sistema                                                         \\\hline
        2a.1               & Elimina una ricetta o una preparazione dall’ elenco dei compiti & Registra l’eliminazione di preparazioni o ricette al foglio riepilogativo \\\hline
    \end{tabular}
\end{table}

\addcontentsline{toc}{subsection}{Estensioni 5}
\begin{table}[H]\centering\caption*{Estensione 5a}
    \small
    \begin{tabular}{|c|p{7cm}|p{6.23cm}|}
        \hline\bfseries \# & \bfseries Attore                                    & \bfseries Sistema                                                                                   \\\hline
        5a.1               & Specifica che una ricetta/preparazione è già pronta & Registra nel foglio riepilogativo che una ricetta/preparazione in un dato assegnamento è già pronta \\\hline
    \end{tabular}
\end{table}

\begin{table}[H]\centering\caption*{Estensione 5b}
    \small
    \begin{tabular}{|c|p{7cm}|p{6.23cm}|}
        \hline\bfseries \# & \bfseries Attore        & \bfseries Sistema                                     \\\hline
        5b.1               & Elimina un assegnamento & Registra il nuovo foglio con l’assegnamento eliminato \\\hline
    \end{tabular}
\end{table}

\begin{table}[H]\centering\color{red}\caption*{Eccezione 5c}
    \small
    \begin{tabular}{|c|p{7cm}|p{6.23cm}|}
        \hline\bfseries \# & \bfseries Attore                                                                                        & \bfseries Sistema                                 \\\hline
        5c.1               & Assegna un compito specificando cosa (ricetta/preparazione), quando (turno) e opzionalmente chi (cuoco) & \color{red}{Cuoco non disponibile nel dato turno} \\\hline
    \end{tabular}
\end{table}