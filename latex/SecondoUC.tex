\documentclass[letterpaper, italian]{report}
\usepackage[utf8]{inputenc}
\usepackage{csvsimple}
\usepackage[hidelinks, unicode]{hyperref}
\usepackage{graphicx}
\usepackage[T1]{fontenc}
\usepackage{contour}
\usepackage{ulem}
\usepackage{bookmark}
\usepackage{enumitem}
\usepackage{longtable}
\usepackage{booktabs}
\usepackage{float}
\usepackage{xcolor}
% \usepackage[landscape,twocolumn]{geometry} here if we want to make it landscape later

\renewcommand{\ULdepth}{1.8pt}
\contourlength{0.8pt}

\newcommand{\myuline}[1]{%
  \uline{\phantom{#1}}%
  \llap{\contour{white}{#1}}%
}

\renewcommand{\thesection}{\arabic{section}} % removes reference of \chapter to avoid "0."
\renewcommand{\thesubsection}{\thesection.\alph{subsection}} % removes reference of \chapter to avoid "0."
% \titleformat{\chapter}{\normalfont\huge}{\thechapter}{20pt}{\huge\it}

\title{
    \leavevmode{\includegraphics[width=0.8\textwidth]{resources/Universita-degli-studi-di-torino-logo.png}\newline\newline}\\
    Cat \& Ring \\
    \large Gestire i Compiti della Cucina
}
\author{Eduard Antonovic Occhipinti, Riccardo Cardona}

\begin{document}

\maketitle

\tableofcontents

\part{Requisiti}
\chapter{UC Dettagliato}

\section*{Informazioni generali}
\textbf{Nome caso d'uso}{: Gestire le ricette}\newline
\textbf{Portata}{: Sistema}\newline
\textbf{Livello}{: Obiettivo utente}\newline
\textbf{Attore primario}{: Cuoco}\newline
\textbf{Parti Interessate}\newline
\textbf{Pre-condizioni}{: L'attore deve essere autenticato come Cuoco}\newline
\textbf{Garanzie di successo o post-condizioni}{: La procedura di cucina è salvata}

\section*{Scenario Principale di Successo}

\chapter{Modello di Dominio e SSD}
\section{Modello di Dominio}
\section{Diagrammi di Sequenza di Sistema}

\chapter{Contratti delle Operazioni}
\textbf{Pre-condizione generale}{: l'utente è identificato come Chef}\textit{h}

\part{Progettazione}
\chapter{Domain Class Diagram}
\chapter{Design Sequence Diagram}

\chapter{Implementazione}
\end{document}