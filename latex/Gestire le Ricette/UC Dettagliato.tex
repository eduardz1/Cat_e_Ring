
\section*{Informazioni generali}
\textbf{Nome caso d'uso}{: Gestire le ricette}\newline
\textbf{Portata}{: Sistema}\newline
\textbf{Livello}{: Obiettivo utente}\newline
\textbf{Attore primario}{: Cuoco}\newline
\textbf{Parti Interessate}\newline
\textbf{Pre-condizioni}{: L'attore deve essere autenticato come Cuoco}\newline
\textbf{Garanzie di successo o post-condizioni}{: La procedura di cucina è salvata}

\small
\section*{Scenario Principale di Successo}\addcontentsline{toc}{section}{Scenario Principale di Successo}
\def\arraystretch{1.55}%  1 is the default, change whatever you need
\begin{tabular}{|c|p{7cm}|p{6.5cm}|}
      \hline\bfseries \# & \bfseries Attore                                                                                                                 & \bfseries Sistema                                         \\\hline

      1                  & Crea una nuova procedura di cucina passandogli un nome specificando se la procedura è una ricetta o una preparazione             & Registra la nuova procedura e la salva nel ricettario     \\\hline
      2                  & Opzionalmente aggiunge uno o più passi alla procedura e opzionalmente specificando le ripetizioni e se è preparabile in anticipo & Registra i passi della procedura                          \\\hline
      \multicolumn{3}{|r|}{\textit{Se desidera creare una variante di una ricetta torna al passo 1}}                                                                                                                    \\\hline
      3                  & Annota gli ingredienti con opzionalmente le dosi                                                                                 &                                                           \\\hline
      \multicolumn{3}{|r|}{\textit{Ripete finché non è soddisfatto}}                                                                                                                                                    \\\hline
      4                  & Opzionalmente consulta il ricettario                                                                                             & Fornisce il ricettario                                    \\\hline
      \multicolumn{3}{|r|}{\textit{Se desidera creare una variante di una preparazione torna al passo 1}}                                                                                                               \\\hline
      5                  & Opzionalmente usa una preparazione del ricettario come ingrediente della procedura                                               & Registra la preparazione come ingrediente della procedura \\\hline
      6                  & Opzionalmente modifica un passo della procedura,Registra il nuovo passo della procedura                                          &                                                           \\\hline
      7                  & Opzionalmente dettaglia la procedura inserendo le porzioni, la tempistica e annota se può essere preparata in anticipo           & Registra le nuove informazioni alla procedura             \\\hline
      \multicolumn{3}{|r|}{\textit{Se desidera torna al passo 2 altrimenti prosegue}}                                                                                                                                   \\\hline
      8                  & Pubblica la procedura nel ricettario opzionalmente modificando il titolo                                                         & La procedura diventa visibile a tutti nel ricettario      \\\hline
\end{tabular}

\addcontentsline{toc}{subsection}{Estensioni 1}
\begin{table}[H]\caption*{Estensione 1a}
      \small
      \begin{tabular}{|c|p{7cm}|p{6.23cm}|}
            \hline\bfseries \# & \bfseries Attore                                                                & \bfseries Sistema           \\\hline
            1a.1               & Crea una variante di una procedura dandogli un nome riferito a quello originale & Registra la nuova procedura \\\hline
      \end{tabular}
\end{table}

\begin{table}[H]\caption*{Estensione 1b}
      \small
      \begin{tabular}{|c|p{7cm}|p{6.23cm}|}
            \hline\bfseries \# & \bfseries Attore                   & \bfseries Sistema                                                                             \\\hline
            1b.1               & Apri una procedura per modificarla & Fornisce la procedura richiesta rendendola non più visibile a tutti e utilizzabile in un menù \\\hline
      \end{tabular}
\end{table}

\begin{table}[H]\caption*{Estensione 1c}
      \small
      \begin{tabular}{|c|p{7cm}|p{6.24cm}|}
            \hline\bfseries \# & \bfseries Attore                & \bfseries Sistema                    \\\hline
            1c.1               & Elimina una procedura esistente & Elimina una procedura dal ricettario \\\hline
      \end{tabular}
\end{table}

\addcontentsline{toc}{subsection}{\indent\color{red}{Eccezioni 1}}
\begin{table}[H]\centering\color{red}\caption*{Eccezione 1b.1a}
      \small
      \begin{tabular}{|c|p{7cm}|p{5.8cm}|}
            \hline\bfseries \# & \bfseries Attore                   & \bfseries Sistema                                  \\\hline
            1b.1a.1            & Apri una procedura per modificarla & La procedura serve per un menù in uno in un evento \\\hline
            \multicolumn{3}{|r|}{\textit{Termina il caso d'uso}}                                                         \\\hline
      \end{tabular}
\end{table}

\begin{table}[H]\centering\color{red}\caption*{Eccezione 1b.1b}
      \small
      \begin{tabular}{|c|p{7cm}|p{5.8cm}|}
            \hline\bfseries \# & \bfseries Attore                   & \bfseries Sistema                                                                                          \\\hline
            1b.1b.1            & Apri una procedura per modificarla & La procedura non è di proprietà dell’attore che sta cercando di modificarlo pertanto non si può proseguire \\\hline
            \multicolumn{3}{|r|}{\textit{Termina il caso d'uso}}                                                                                                                 \\\hline
      \end{tabular}
\end{table}

\begin{table}[H]\centering\color{red}\caption*{Eccezione 1c.1a}
      \small
      \begin{tabular}{|c|p{7cm}|p{5.8cm}|}
            \hline\bfseries \# & \bfseries Attore                & \bfseries Sistema                                  \\\hline
            1c.1a.1            & Elimina una procedura esistente & La procedura serve per un menù in uno in un evento \\\hline
            \multicolumn{3}{|r|}{\textit{Termina il caso d'uso}}                                                      \\\hline
      \end{tabular}
\end{table}

\begin{table}[H]\centering\color{red}\caption*{Eccezione 1c.1b}
      \small
      \begin{tabular}{|c|p{7cm}|p{5.8cm}|}
            \hline\bfseries \# & \bfseries Attore                & \bfseries Sistema                                                                                         \\\hline
            1c.1b.1            & Elimina una procedura esistente & La procedura non è di proprietà dell’attore che sta cercando di eliminarlo pertanto non si può proseguire \\\hline
            \multicolumn{3}{|r|}{\textit{Termina il caso d'uso}}                                                                                                             \\\hline
      \end{tabular}
\end{table}

\addcontentsline{toc}{subsection}{Estensioni 2}
\begin{table}[H]\centering\caption*{Estensione 2a}
      \small
      \begin{tabular}{|c|p{7cm}|p{6.24cm}|}
            \hline\bfseries \# & \bfseries Attore                                                                                                                                                                   & \bfseries Sistema                                         \\\hline
            2a.1               & Opzionalmente crea un raggruppamento di passi dati uno o più passi semplici da inserire nella procedure e opzionalmente specificando le ripetizioni e se è preparabile in anticipo & Registra il nuovo raggruppamento di passi nella procedura \\\hline
            \multicolumn{3}{|r|}{\textit{Se desidera creare una variante di una ricetta torna al passo 1}}                                                                                                                                                                      \\\hline
      \end{tabular}
\end{table}

\begin{table}[H]\centering\caption*{Estensione 2b}
      \small
      \begin{tabular}{|c|p{7cm}|p{6.24cm}|}
            \hline\bfseries \# & \bfseries Attore                                                                                                                                 & \bfseries Sistema                                          \\\hline
            2b.1               & Opzionalmente crea una variante di un passo inserendo 2 passi da inserire nella procedura e opzionalmente indicando se è preparabile in anticipo & Registra il nuovo raggruppamento di passi per la procedura \\\hline
            \multicolumn{3}{|r|}{\textit{Se desidera creare una variante di una ricetta torna al passo 1}}                                                                                                                                     \\\hline
      \end{tabular}
\end{table}

\addcontentsline{toc}{subsection}{Estensioni 3}
\begin{table}[H]\centering\caption*{Estensione 3a}
      \small
      \begin{tabular}{|c|p{7cm}|p{6.24cm}|}
            \hline\bfseries \# & \bfseries Attore                                   & \bfseries Sistema                                             \\\hline
            3a.1               & Modifico le dosi di un ingrediente della procedura & Registro la procedura con le dosi dell’ingrediente modificato \\\hline
      \end{tabular}
\end{table}

\begin{table}[H]\centering\caption*{Estensione 3b}
      \small
      \begin{tabular}{|c|p{7cm}|p{6.24cm}|}
            \hline\bfseries \# & \bfseries Attore                       & \bfseries Sistema                     \\\hline
            3b.1               & Elimino un ingrediente della procedura & Elimino l’ingrediente dalla procedura \\\hline
      \end{tabular}
\end{table}

\addcontentsline{toc}{subsection}{Estensioni 5}
\begin{table}[H]\centering\caption*{Estensione 5a}
      \small
      \begin{tabular}{|c|p{7cm}|p{6.24cm}|}
            \hline\bfseries \# & \bfseries Attore                       & \bfseries Sistema                                            \\\hline
            5a.1               & Elimino una procedura come ingrediente & Elimino la procedura come ingrediente della procedura creata \\\hline
      \end{tabular}
\end{table}

\addcontentsline{toc}{subsection}{Estensioni 6}
\begin{table}[H]\centering\caption*{Estensione 6a}
      \small
      \begin{tabular}{|c|p{7cm}|p{6.24cm}|}
            \hline\bfseries \# & \bfseries Attore                                          & \bfseries Sistema                                  \\\hline
            6a.1               & Opzionalmente modifico l’ordine dei passi della procedura & Registra il nuovo ordine dei passi nella procedura \\\hline
      \end{tabular}
\end{table}

\begin{table}[H]\centering\caption*{Estensione 6b}
      \small
      \begin{tabular}{|c|p{7cm}|p{6.24cm}|}
            \hline\bfseries \# & \bfseries Attore                               & \bfseries Sistema                \\\hline
            6b.1               & Opzionalmente elimino un passo della procedura & Elimino il passo della procedura \\\hline
      \end{tabular}
\end{table}

\addcontentsline{toc}{subsection}{Estensioni 7}
\begin{table}[H]\centering\caption*{Estensione 7a}
      \small
      \begin{tabular}{|c|p{7cm}|p{6.24cm}|}
            \hline\bfseries \# & \bfseries Attore                                  & \bfseries Sistema                               \\\hline
            7a.1               & Opzionalmente modifica i dettagli della procedura & Registra la procedura con i dettagli modificati \\\hline
      \end{tabular}
\end{table}

\begin{table}[H]\centering\caption*{Estensione 7b}
      \small
      \begin{tabular}{|c|p{7cm}|p{6.24cm}|}
            \hline\bfseries \# & \bfseries Attore                                      & \bfseries Sistema                     \\\hline
            7b.1               & Opzionalmente modifica il tag di una annotazione della procedura & Registra la modifica del tag di una annotazione \\\hline
      \end{tabular}
\end{table}

\begin{table}[H]\centering\caption*{Estensione 7c}
      \small
      \begin{tabular}{|c|p{7cm}|p{6.24cm}|}
            \hline\bfseries \# & \bfseries Attore                                      & \bfseries Sistema                     \\\hline
            7c.1               & Opzionalmente elimina una annotazione dalla procedura & Elimina l’annotazione dalla procedura \\\hline
      \end{tabular}
\end{table}

\begin{table}[H]\centering\caption*{Estensione 7d}
      \small
      \begin{tabular}{|c|p{7cm}|p{6.24cm}|}
            \hline\bfseries \# & \bfseries Attore                                                                                          & \bfseries Sistema                                     \\\hline
            7d.1               & Crea una procedura passando un nome e specificando se è una ricetta o una preparazione      & Registra la nuova procedura e la salva nel ricettario \\\hline
            7d.2               & Opzionalmente seleziono uno o più passi da inserire nella procedura appena creata                         & Registra i passi della procedura                      \\\hline
            7d.3               & Opzionalmente seleziono uno o più ingredienti con relative dosi da inserire nella procedura appena creata & Registra gli ingredienti della procedura              \\\hline
      \end{tabular}
\end{table}

\addcontentsline{toc}{subsection}{Estensioni 8}
\begin{table}[H]\centering\caption*{Estensione 8a}
      \small
      \begin{tabular}{|c|p{7cm}|p{6.24cm}|}
            \hline\bfseries \# & \bfseries Attore                                 & \bfseries Sistema \\\hline
            8a.1               & Conclude il lavoro senza pubblicare la procedura &                   \\\hline
      \end{tabular}
\end{table}
\normalsize